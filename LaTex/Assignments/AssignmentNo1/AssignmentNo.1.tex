\documentclass[12pt]{article}

% Packages 
\usepackage{geometry}
\geometry{margin=1in}

\usepackage{xcolor}
\usepackage{setspace}
\usepackage{titlesec}

\definecolor{titlecolor}{RGB}{0,51,102}
\definecolor{sectioncolor}{RGB}{0,76,153}
\definecolor{subsectioncolor}{RGB}{64,64,64}

\titleformat{\section}
  {\color{sectioncolor}\bfseries\Large}
  {\thesection}{1em}{}

\titleformat{\subsection}
  {\color{subsectioncolor}\bfseries\large}
  {\thesubsection}{1em}{}

\title{
\color{titlecolor}
\textbf{Programming: Teaching Machines How to Think and Humans How to Solve Problems}
}

\author{
\textbf{Bhavesh Gadling} \\
MIS Number: 612415061 \\
Affiliation: COEP Technological University
}

\date{13 January 2025}
% document start
\begin{document}
\maketitle
\pagenumbering{gobble}

% Introduction 
\section{Introduction}

\paragraph{} Programming is often misunderstood as the simple act of writing code. 
In reality, it is the art of breaking complex ideas into precise instructions that a machine can understand.
A programmer does not merely type commands; instead, they design logic, anticipate failures, and transform abstract problems into working systems.
Thus, programming is less about computers and more about how humans think and reason.

% Problem Solving
\section{Programming as a Problem-Solving Tool}

\subsection{Understanding the Problem}

\paragraph{} At its core, programming is problem-solving.
Every program begins with a fundamental question: \textit{How can a task be completed efficiently and correctly?}
Whether the task involves performing calculations, managing large datasets, or operating digital platforms,
the programmer must analyze the problem carefully before attempting to solve it.

\subsection{Breaking Problems into Components}

\paragraph{} A key aspect of programming is dividing complex problems into smaller, manageable parts.
These components are then arranged in a logical sequence that a computer can execute.
This structured approach not only improves program efficiency but also enhances clarity of thought,
allowing programmers to handle even the most complex challenges with confidence.

% Precision
\section{The Importance of Precision}

\paragraph{} One of the most fascinating aspects of programming is its demand for absolute precision.
Computers follow instructions exactly as they are written—nothing more and nothing less.
Even a minor error, such as a missing symbol or an incorrect condition, can cause an entire system to fail.

\subsection{Learning Through Debugging}

\paragraph{} Debugging plays a crucial role in the programming process.
It teaches patience, discipline, and attention to detail.
Rather than viewing errors as failures, programmers learn to treat them as opportunities to refine their logic and improve their solutions.

% Creativity
\section{Creativity in Programming}

\paragraph{} Despite its logical foundation, programming is also a creative discipline.
There are often multiple ways to solve the same problem, and each programmer may choose a different algorithm, structure, or design approach.
From developing games and websites to building complex software systems,
programming enables individuals to transform imagination into functional reality.} 

% Impact
\section{Impact of Programming on the Modern World}

\paragraph{} Programming plays a vital role in shaping the modern world.
It powers banking systems, healthcare platforms, communication networks, and space exploration technologies.
Through automation, software allows tasks to be completed faster, more accurately, and at an unprecedented scale.
A single well-designed program can serve millions of users worldwide.

% Conclusion
\section{Conclusion}

\paragraph{} In conclusion, programming is far more than a technical skill—it is a way of thinking.
It fosters logical reasoning, creativity, and resilience in the face of challenges.
By learning how to instruct machines, programmers also learn how to approach problems systematically and confidently.
In an increasingly software-driven world, programming stands as one of the most powerful and transformative skills a person can acquire.

\end{document}
