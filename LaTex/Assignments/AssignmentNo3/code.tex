% Compile with XeLaTeX
\documentclass[dvipsnames,mathserif,aspectratio=169,10pt]{beamer}

\setbeamertemplate{footline}[frame number]
\setbeamercolor{footline}{fg=black}
\setbeamerfont{footline}{series=\bfseries}

\usepackage{tikz}
\usepackage{xcolor}
\usepackage{ragged2e}
\usepackage{amsmath}

\usetheme{Frankfurt}

\setbeamertemplate{navigation symbols}{}

% LOGO
\logo{%
  \makebox[0.98\paperwidth]{
    \includegraphics[width=0.80cm]{Logos/COEP-Tech-Pune-Logo-3013335431.png}
    \hspace{5cm}
    \includegraphics[scale=0.20]{Logos/co.png}
    \hfill
    \includegraphics[]{Logos/Logo.png}
  }
}

% Item style
\setbeamertemplate{itemize item}{\scriptsize\raise1.25pt\hbox{$\blacktriangleleft$}}

% Title info (from IEEE paper)
\title[Programming Concepts]
{An Overview of Programming Concepts and Their Applications}

\author[Bhavesh Gadling]
{Bhavesh Gadling \\ MIS: 612415061}

\institute[COEP Tech]
{
Department of Computer Science and Engineering\\
COEP Technological University\\
Pune, India
}

\date{\today}

\begin{document}

% Title Slide
\begin{frame}
\maketitle
\end{frame}

% Table of Contents
\begin{frame}{Outline}
\begin{center}
\footnotesize
\tableofcontents
\end{center}
\end{frame}


% Introduction
\section{Introduction}

\begin{frame}{Introduction}
\begin{itemize}
    \item Programming is the process of designing and writing instructions for computers.
    \item Programs are written using formal programming languages.
    \item Enables problem-solving from simple tasks to complex systems.
    \item Forms the foundation of modern computing.
\end{itemize}
\end{frame}

% Evolution of Programming
\section{Evolution of Programming}

\begin{frame}{Evolution of Programming Languages}
\begin{itemize}
    \item Early programs used machine and assembly languages.
    \item Difficult to write, understand, and debug.
    \item Introduction of high-level languages improved abstraction.
    \item Examples: C, Java, Python.
\end{itemize}
\end{frame}

% Methodology
\section{Methodology}
\begin{frame}{Methodology}
\begin{enumerate}
    \item Identification of fundamental programming concepts
    \item Study of programming paradigms
    \item Mathematical analysis of algorithms
    \item Comparative analysis of programming languages
\end{enumerate}
\end{frame}

\begin{frame}{Algorithm Performance Analysis}
Algorithm execution time is represented by Equation~(1):

\begin{equation}
T(n) = n^2 + n + 1
\end{equation}

\begin{itemize}
    \item Equation~(1) shows growth with input size
    \item Used to estimate algorithm efficiency
\end{itemize}
\end{frame}

%  FIGURE
\begin{frame}{Applications of Programming}
\begin{itemize}
    \item Programming impacts multiple computing domains
    \item Figure~\ref{fig:apps} illustrates major application areas
\end{itemize}

\begin{figure}
\centering
\includegraphics[width=0.28\textwidth]{programming_domains}
\caption{Major Application Areas of Programming}
\label{fig:apps}
\end{figure}
\end{frame}

% TABLE 
\begin{frame}{Programming Language Comparison}
\begin{itemize}
    \item Languages differ based on paradigm and usage
    \item Table~\ref{tab:lang} compares common programming languages
\end{itemize}

\begin{table}
\centering
\begin{tabular}{|c|c|c|}
\hline
Language & Paradigm & Application Area \\
\hline
C & Procedural & System Programming \\
Java & Object-Oriented & Application Development \\
Python & Multi-Paradigm & Data Science \\
\hline
\end{tabular}
\caption{Comparison of Programming Languages}
\label{tab:lang}
\end{table}
\end{frame}

% Applications of Programming 
\section{Applications of Programming}
\begin{frame}{Applications of Programming}
\begin{itemize}
    \item Software and application development
    \item Operating systems
    \item Web and cloud computing
    \item Embedded systems and IoT
    \item Artificial intelligence and data science
\end{itemize}
\end{frame}
% Programming in modern systems
\section{Programming in Modern Systems}

\begin{frame}{Programming in Modern Computing Systems}
\begin{itemize}
    \item Core component of operating systems.
    \item Enables cloud scalability and automation.
    \item Critical in AI and real-time systems.
    \item Supports distributed and high-performance computing.
\end{itemize}
\end{frame}

% Comparative Study
\section{Comparative Study}

\begin{frame}{Comparison of Programming Languages}

\begin{itemize}
    \item Languages differ based on paradigm and usage
    \item Table~\ref{tab:lang} compares common programming languages
\end{itemize}

\begin{table}
\centering
\begin{tabular}{|c|c|c|}
\hline
Language & Paradigm & Application Area \\
\hline
C & Procedural & System Programming \\
Java & Object-Oriented & Application Development \\
Python & Multi-Paradigm & Data Science \\
\hline
\end{tabular}
\caption{Comparison of Programming Languages}
\label{tab:lang}
\end{table}

\end{frame}

% Results
\section{Results}

\begin{frame}{Results}
\begin{itemize}
    \item Strong fundamentals improve problem-solving efficiency.
    \item Algorithm performance depends on input size.
    \item Different languages suit different application domains.
\end{itemize}
\end{frame}

% Conclusion
\section{Conclusion}

\begin{frame}{Conclusion}
\begin{itemize}
    \item Programming is essential in modern engineering.
    \item Understanding concepts and paradigms is critical.
    \item Enables efficient and scalable software solutions.
    \item Continuous learning is necessary as technology evolves.
\end{itemize}
\end{frame}

% Reference
\section{References}

\begin{frame}{References}
\footnotesize
\begin{itemize}
    \item Kernighan \& Ritchie, \textit{The C Programming Language}
    \item Herbert Schildt, \textit{Java: The Complete Reference}
    \item IEEE Author Guidelines
\end{itemize}
\end{frame}

\end{document}

